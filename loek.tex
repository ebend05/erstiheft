\chapter{Bachelor of Science (B.Sc.) Landschaftsökologie}
\lohead{\footnotesize{Geographie - \textbf{Landschaftsökologie} - Geoinformatik}}
\rehead{\footnotesize{Geographie - \textbf{Landschaftsökologie} - Geoinformatik}}

\section{Allgemeines}
Landschaftsökologie?! ...von diesem Studienfach haben viele Leute noch nie gehört. Was wird man denn, wenn man das studiert und was ist das überhaupt? Dies sind Fragen, die man als angehender Landschaftsökologe wohl noch sehr oft zu hören bekommen wird. Was euch genau erwartet, wie eure Prüfungsordnung im Detail aussieht und was am ILÖK sonst noch so los ist, könnt ihr auf den folgenden Seiten nachlesen.

\subsection*{Bewerbung}
Die Bewerbung erfolgt direkt über die Universität Münster (nicht über die ZVS). Es stehen zurzeit etwa 70 Studienplätze zur Verfügung. Anfangen kann man nur im Wintersemester. Da einige Studierende ihren Studienplatz nicht antreten, gibt es in der Regel mehrere Nachrückverfahren. Außerdem gibt es auch immer einige, die während der ersten Semester das Studium abbrechen oder das Fach wechseln.

\section{Studienaufbau}
Den Studienaufbau sowie Informationen zu ALLEM findet ihr in eurer Prüfungsordnung unter \ https://www.uni-muenster.de/Landschaftsoekologie/studieren/bsc_loek.html. \ Ihr studiert nach der neuen Studienordnung von 2016, schaut bei Gelegenheit einfach mal auf der Homepage vorbei. Dieses Erstiheft dient nur als Information, die Prüfungsordnung schreibt fest, was ihr für ein erfolgreiches Studium tun müsst. 
Die Studienordnung für den Bachelor Landschaftsökologie ist „modularisiert“ aufgebaut. Die Regelstudienzeit beträgt 3 Jahre, also 6 Semester. Das Bachelorstudium besteht aus insgesamt 27 Modulen, wovon 26 studiert werden müssen, und schließt ein Berufspraktikum, ein Studienprojekt und die Bachelorarbeit ein. Insgesamt müssen für den Bachelorabschluss 180 Credit Points (ECTS) erreicht werden. Für die meisten Module gibt es 5 bis 15 Credit Points. Ein Credit Point entspricht einer Arbeitszeit von ca. 30 Stunden (inklusive Kontakt"= und Präsenzzeiten, Vorbereitung, Nachbereitung und Prüfungsvorbereitung). 

Module bestehen aus inhaltlich aufeinander abgestimmten oder sich ergänzenden Lehrveranstaltungen. Die Module erstrecken sich in der Regel über zwei aufeinander folgende Semester, also ein Jahr. Die einzelnen Veranstaltungen eines Moduls werden abgeprüft. Dazu gibt es grundsätzlich mehrere Möglichkeiten, z. B. mündliche Prüfung, Klausur, Test, Vorträge, Protokolle usw. In welchem Modul wie viele oder welche Leistungsnachweise erbracht werden müssen, kann sehr unterschiedlich sein und hängt auch vom Dozenten ab. Fest steht, dass jedes Modul abgeprüft werden muss. Weitere Infos dazu findet ihr in der Prüfungsordnung. Dort steht auch, welche dieser Leistungsnachweise relevant für die Abschlussnote sind.

Vor oder bei der ersten Sitzung jeder Veranstaltung müssen die Lehrenden bekannt geben, welchen Leistungsnachweis sie verlangen. Im Zweifel fragt direkt nach! Aus diesem Grund ist generell wichtig, gerade in der ersten Veranstaltung anwesend zu sein. Hier werden außerdem Termine festgelegt, Plätze und ggf. Referatsthemen vergeben und der Zugang zu online hinterlegten Präsentationen und Informationen herausgegeben. 
Die meisten Noten, die in den Modulen erbracht werden, gehen unterschiedlich gewichtet in die Bachelornote ein. Einige Module (zum Beispiel die naturwissenschaftlichen Grundlagenfächer Mathe und Physik) müssen nur bestanden werden und gehen nicht in die Abschlussnote mit ein. Andere Fächer, die für das Studium der Landschaftsökologie von besonderer Relevanz sind, gehen sogar doppelt gewichtet in die Bachelornote ein. Zu diesen Fächern gehören meist die Fächer, die auch mehr Credit Points geben wie Zoologische Formenkenntnis und Tierökologie, Landschaft und Lebensräume und Raum- und Umweltplanung. Auch das steht alles in der Prüfungsordnung... Merkst du was? Doch bevor du jetzt die Prüfungsordnung direkt zu deiner allabendlichen Bettlektüre machst, lies erst mal entspannt weiter...

\subsection*{Definition}
Die Landschaftsökologie beschäftigt sich mit der Beziehung zwischen belebter und unbelebter Umwelt, wobei qualitative und quantitative Analysen von Ökosystemen und deren Lebensgemeinschaften im Zentrum der Betrachtung stehen. Es werden die Stoff- und Energiekreisläufe in ihnen und die Wechselbeziehungen untereinander in räumlicher und zeitlicher Entwicklung betrachtet.

Unser Institut besitzt fachliche Schwerpunkte auf den Gebieten der Angewandten Landschaftsökologie/Ökologische Planung, Biodiversität und Ökosystemforschung, Bodenökologie und Landnutzung, Klimatologie, Ökohydrologie und Stoffkreisläufe, Tierökologie und multitrophische Interaktionen sowie Waldökologie, Forst- und Holzwirtschaft. Dies sind auch die sieben Arbeitsgruppen am ILÖK. Im Vergleich zur Geoökologie, die stärker abiotisch ausgerichtet ist, gibt es bei der Landschaftsökologie in etwa ein Gleichgewicht zwischen abiotischem und biotischem Themenspektrum (biotisch = belebte Natur; abiotisch = unbelebte Natur z. B. Gestein, Wasser-Chemie, etc.).

\section*{Historisches}
Ab 1987 konnte man in Münster am Institut für Geographie Diplom"=Geographie mit den Studienschwerpunkten Landschaftsökologie (LÖK) oder Sozialgeographie (SOZ) studieren. Das Lehrangebot war dem heutigen in weiten Teilen ähnlich. In jenen Tagen gab es noch keinen NC, und somit war der Andrang an Studierenden groß. Für die Landschaftsökologie−Geos bedeutete dies, dass nicht alle Studierenden das volle Lehrangebot nutzen konnten (z. B. Geländepraktika, Hauptseminare), sondern nur beschränkt auszuwählen vermochten.

1992 wurde dann das Große Schwert gezückt, und der alte Studiengang Geographie wurde in zwei völlig unterschiedliche Teile geschlagen: (Sozial-) Geographie und Landschaftsökologie als eigenständige Diplom"=Studiengänge. Jetzt kam auch Meister Numerus Clausus vorbei, um der BewerberInnen"=Flut Herr zu werden. 1994 ist das alte Institut für Geographie (IfG) aufgelöst worden und es entstanden das (neue) Institut für Geographie (IfG), das Institut für Geoinformatik (IfGI) und das Institut für Landschaftsökologie (IfL, heute ILÖK), sowie eine verbindende Betriebseinheit für die Verwaltung.
An der Uni Münster blickt man daher schon auf eine gewisse landschaftsökologische Tradition zurück. Der Studienverlauf hat sich mit den Jahren einige Male geändert und das Diplom Landschaftsökologie ist inzwischen durch die neuen Studiengänge Bachelor und Master ersetzt worden.

\section{Modulübersicht}
\begin{longtable}{p{0.75\textwidth} p{0.1\textwidth} p{0.1\textwidth}}
 & Typ & ECTS \\ 
\textbf{B1 Geologie} & & \textbf{5}\\ 
Einführung in die Geoogie und Geomorphologie & V & 2\\
Gesteinskunde & Ü & 3\\
& &\\
\textbf{B2 Bodenkunde}& & \textbf{5}\\
Einführung in die Bodenkunde & V & 2\\
Geländepraktikum Boden & P & 3\\
& &\\
\textbf{B3 Allgemeine Biologie}& & \textbf{5}\\
Biologie II & V &5\\
& &\\
\textbf{B4 Botanische Formenkenntnis} & & \textbf{5}\\
Bestimmungsübungen Botanik & Ü & 5\\
& &\\
\textbf{B5 Zoologische Formenkenntnis und Tierökologie} & &\textbf{10}\\
Enführung in die Tierökologie & V & 2\\
Systematik der Tiere & Ü & 2\\
Zoologische Bestimmungsübung & Ü & 6\\
& &\\
\textbf{B6 Chemie für Naturwissenschaftler} & & \textbf{10}\\
Chemie für Naturwissenschaftler & V & 4\\
Übung Chemie & Ü & 2\\
Einführungspraktikum & P & 4 \\
&&\\
\textbf{B7 Mathematik} & &\textbf{5}\\
Mathe für Naturwissenschaftler & V & 2\\
Übung Mathematik & Ü & 3\\
&&\\
\textbf{B8 Physik} & & \textbf{5}\\
Physik für Naturwissenschaftler & V & 3 \\
Experimentalphysik für Löks & Ü & 2\\
&&\\
&&\\
&&\\
\textbf{B9 Vegetationsökologie} & & \textbf{5}\\
Einführung in die Vegetationsökologie & V & 2 \\
Geländeübung Vegetationsökologie & Ü & 3\\
&&\\
\textbf{B10 Exkursionen}&&\textbf{8}\\
Mindestens 8 Tage plus begleitendes Seminar (Wahlpflicht) & Exk. & 8\\
Oder mindestens 12 Tage (Wahlpflicht) & Exk. & 8\\
&&\\
\textbf{B11 Allgemeine Studien I (Arbeitstechniken)}& & \textbf{5}\\
Studien- und Arbeitstechniken & S & 2\\
Tutorium zu Studien- und Arbeitstechniken & T & 1\\
Fachenglisch & S & 1\\
Berufliche Orientierung & S & 1\\
&&\\
\textbf{B12 Allgemeine Studien II (Projektmanagement)}&& \textbf{5}\\
ersetzbar durch Modul B22 (Ergänzungsmodul III)\\ 
Grundlagen des Projektmanagements & S & 2\\
Praxisprojekt & P & 2\\
Begleitseminar & S & 1 \\
&&\\
\textbf{B13 Klimatologie} && \textbf{10}\\
Einführung in die Klimatologie & V & 2\\
Übung Klimatologie & Ü & 3\\
Klimatologie II & V & 2 \\
Seminar Klimatologie & S & 3 \\
&&\\
\textbf{B14 Wasser- und Stoffkreisläufe } && \textbf{10}\\
Einführung in die Hydrologie & V & 2\\
Einführung in die Wasserchemie &V&1\\
Einführung in die Biogeochemie & V & 1 \\
Übung Wasser- und Stoffhaushalt & Ü & 3 \\
Laborpraktikum Wasser- und Biogeochemie & P & 3 \\

&&\\
\textbf{B16 Landschaften und Lebensräume} & & \textbf{10}\\
Ökosysteme und Lebensgemeinschaften & V & 2\\
Landschaftszonen der Erde & V & 2\\
Landschaftsökologische Übung & Ü & 6\\
&&\\
\textbf{B17 Geostatistik} && \textbf{5}\\
Geostatistik & V & 2\\
Übung Geostatistik & Ü & 3\\
&&\\
\textbf{B18 Geoinformatik} && \textbf{10}\\
Einführung in die Geoinformatik & V & 2\\
Geoinformatik für Geowissenschaftler & Ü & 3\\
Angewandte Karthographie & Ü & 5\\
&&\\
\textbf{B19 Methoden der Landschaftserfassung}&&\textbf{5}\\
Einführung in die Fernerkundungsmethoden in den Geowissenschaften (Pflicht) & V & 2\\
Fernerkundungsmethoden in den Geowissenschaften (Wahlpflicht) & Ü & 3\\
GPS-Methoden & Ü & 3\\
Biotop- und FFH-Lebensraumkartierung (Wahlpflicht) & V + Ü & 3\\
Wissenschaftliches Rechnen (Wahlplicht) & Ü & 3\\
Tierökologische Erfassungsmethoden (Wahlplicht) & Ü & 3\\
GIS-Grundkurs (Wahlplicht)  & Ü & 3\\
evtl. weitere Angebote\\
&&\\
\textbf{B20 Ergänzungsmodul I} && \textbf{5}\\
Verschiedene Veranstaltungen & & 5\\
&&\\
\textbf{B21 Ergänzungsmodul II} && \textbf{5}\\
Verschiedene Veranstaltungen & & 5\\
&&\\
\textbf{B22 Ergänzungmodul III} && \textbf{5}\\
möglicher Ersatz für B12 (Allgemeine Studien II (Projektmanagement))\\
Verschiedene Veranstaltungen & & 5\\
&&\\
\textbf{B23 Raum- und Umweltplanung}&&\textbf{10}\\
Grundlagen der Raumplanung & V & 2\\
Grundlagen der Raumplanung & Ü & 3\\
Grundlagen der Ökologischen Planung & V & 2\\
Grundlagen der Ökologischen Planung & Ü & 3\\
&&\\
\textbf{B24 Angewandte Landschaftsökologie} && \textbf{10}\\
Studienprojekt & & 10\\
&&\\
\textbf{B25 Berufsorientiertes Praktikum} & & \textbf{10}\\
Berufsorientiertes Praktikum (6 Wochen) \\
&&\\
\textbf{B26 Wissenschaftliches Arbeiten} & & \textbf{5}\\
Projekt- und Teamarbeit & S & 1\\
Projekt-/Studienarbeit & P & 4\\
&&\\
\textbf{B27 Bachelorarbeit} && \textbf{12}\\
&&\\
\textbf{Einen Ersti-Stundenplan findet ihr auf unserer Fachschaftsseite! :)}
\end{longtable}

\subsection*{Naturwissenschaftliche Grundlagen}
\subsubsection*{Mathematik und Physik}
...sind nur jeweils ein Semester lang als Pflichtkurse zu erledigen. Ein Mathe"=Modul (Vorlesung+Übung+Hausaufgaben+Klausur) steht dann schon im 1. Semester an. Ebenso darf man sich im 1. Semester über Physik freuen (Vorlesung+Übung+Klausur). Die Physikübungen werden als Block in den Semesterferien angeboten und sind wirklich keine Hürde. Die Physik-„Klausur“ kommt getarnt als Multiple-Choice-Test mit Kopfrechnen daher – wir erinnern uns an die theoretische Führerscheinprüfung. Die Grundlagen, die man hier lernt, können einem im weiteren Studium aber immer mal wieder begegnen.

\subsubsection*{Chemie}
...ist natürlich auch mit dabei. Hier sind die Vorlesungen und eine Übung im Lehrplan vorgesehen. Eine Eingangsklausur entscheidet über die Zulassung zum Praktikum; dieses findet in den Winter- oder Sommersemesterferien statt. Den Abschluss des zweiwöchigen, ganztägigen Praktikums krönt eine Abschlussklausur. In der Chemievorlesung wird viel Stoff behandelt und die Klausuren sind sehr anspruchsvoll. Bleibt daher bitte am Stoff dran und fangt \textbf{frühzeigig} mit dem Lernen an! Später werden die Kenntnisse im neuen Modul Wasser- und Stoffkreisläufe vertieft und spezialisiert.

\subsubsection{Biologische Grundlagen}
Zunächst gibt es eine allgemeine Vorlesung gemeinsam mit Biologie-Studis, nämlich Biologie II. In der Vorlesung stehen ein Überblick über das Tier"= und Pflanzenreich, das Thema Evolution und ein wenig Verhaltensbiologie auf dem Programm. Das Modul wird mit einer Multiple Choice-Klausur abgeschlossen. Die Vorlesung Biologie I muss von den LÖK-Studis nicht belegt werden, da es dort vornehmlich um Genetik und Zellbiologie geht. Solch kleine, komplizierte Dinge sind sicher interessant, aber für uns Löks (zum Glück?!) meist nicht so relevant...

Hinzu kommen die botanischen und zoologischen Bestimmungsübungen, in denen ihr Pflänzchen und Krabbeltiere unter die Lupe nehmt. In der Zoologie stehen ökologisch relevante Tiergruppen wie zum Beispiel Vögel und Amphibien auf dem Programm. In der Botanik werden die 10 bis 15 wichtigsten Pflanzenfamilien durchgenommen. Man lernt in den Übungen, wie man beim Bestimmen vorgeht, die Erweiterung und das Training der eigenen Artenkenntnisse sind überwiegend Teil des Selbststudiums. Zur botanischen Bestimmungsübung gehört das Anlegen eines Herbariums.

\subsubsection*{Landschaftsökologische Grundlagen}
Diese meist physiogeographisch und ökologischen Grundlagen werden durch verschiedene einführende Vorlesungen vermittelt, die von Vegetations"= und Tierökologie über Bodenkunde, Klimatologie und Hydrologie bis hin zur Landschaftsökologie reichen (siehe Übersicht). Diese Module stellen den Kern des Studiums dar. Zu den Vorlesungen finden meist im darauf folgenden Sommersemester (oder auch parallel) Geländepraktika bzw. Übungen statt, in denen bestimmte praktische Anwendungen dieser Teildisziplinen vorgestellt und von den Studierenden selbst durchgeführt werden. Eure Untersuchungsergebnisse fasst ihr in Form von Protokollen zusammen. Meist bringen die Geländeübungen viel Spaß und eine gute Abwechslung zum Hörsaal oder Computerraum!

\subsubsection*{Geoinformatik und Geostatistik}
Grundlage bildet die Vorlesung \enquote{Einführung in die Geoinformatik} mit der dazugehörigen Übung. Zusätzlich ist dann noch die Übung \enquote{Angewante Kartographie}. Weiterer Bestandteil aus der Geoinformatik ist die Vorlesung „Geostatistik“ mit der dazugehörigen Übung. Alle Übungen werden am Computer durchgeführt - aber keine Sorge: sie sind alle kein Hexenwerk und gut machbar.

\subsection*{Wahlmöglichkeit}
Obwohl der Bachelorstudiengang wesentlich durchstrukturierter ist und weniger Wahlmöglichkeiten als der Diplomstudiengang lässt, geben zwei bzw. drei Ergänzungsmodule die Möglichkeit sich je nach Interessenlage mit anderen Fächern zu beschäftigen und eigene Initiative zu zeigen. Diese sollten dennoch einen Bezug zur Landschaftsökologie haben und daher mit den DozentInnen des ILÖK abgesprochen werden. Diese können euch auch beraten, wie sinnvoll eure Wahl des Ergänzungsmoduls ist. Für jedes der drei Ergänzungsmodule à 5 Leistungspunkte kann man sich ein Fach aussuchen. Wichtig ist hierbei mit den Ergänzungsmodulen rechtzeitig (am besten ab dem 3. Fachsemester) anzufangen, da man für einige Module mehrere Semester einplanen muss. Eine Übersicht über mögliche Ergänzungsmodule findet ihr auf der Seite vom Front Office unter

\url{http://www.uni-muenster.de/Frontoffice-geoloek/Bsc_LOEK_PO.html/}

\Wenn ihr euch unsicher seid, kommt doch mal in der Fachschaft, im Front Office oder bei eurer/m FachstudienberaterIn vorbei und lasst euch beraten. Beispiele für wählbare Module sind Rechtswissenschaften, Politikwissenschaften, Pädagogik/ Erziehungswissenschaften, Wirtschaftswissenschaften, Geoinformatik, Geographie und ggf. Sprachkurse. Die Lehrveranstaltungen variieren je nach Modul und sind ebenso wie die Prüfungsmodalitäten und die Punkte-Vergabe vom jeweiligen Fach festzulegen. Aus organisatorischen Gründen stehen einige Veranstaltungen jedoch nicht in HISLSF.

\subsubsection*{Angewandte Landschaftsökologie}
Das Studienprojekt ist eine Gruppenarbeit (à ca. 10 Personen). Sie soll auf die Bachelorarbeit vorbereiten, an selbständiges Arbeiten heranführen und gleichzeitig das Arbeiten im Team schulen. So ist es beispielsweise möglich, dass sich die Studierenden durch die praktische Bearbeitung von Teilflächen an der Erstellung von Umweltverträglichkeitsprüfungen beteiligen, indem sie floristische und faunistische Kartierungen, Boden- oder Wasseruntersuchungen und klimatologische Messungen vornehmen, um das jeweilige Gebiet charakterisieren zu können. Die Arbeitsgruppen des ILÖK bieten jährlich verschiedene Studienprojekte an, aber wenn ihr möchtet, könnt ihr auch bei verwandten Fachbereich nach Projekten fragen (das bitte vorher mit dem ILÖK absprechen). Die Themen werden im Laufe des vierten Semesters vorgestellt. Das Studienprojekt unterscheidet sich deutlich von den anderen Veranstaltungen und fordert u.a. etwas soziales Geschick, also TEAM-Work.

\subsection*{Berufsorientiertes Praktikum}
Es sollen (mindestens) 6 Wochen ganztägiges, außeruniversitäres Praktikum geleistet werden. Du kannst das Praktikum zum Beispiel in der Verwaltung (kommunal, landesweit usw.), Planungsbüros, Biostationen oder sonstigen Unternehmen der freien Wirtschaft machen. Die Stelle sollst du dir selber suchen, aber vielleicht findest du auch wen, der dir eine gute Stelle empfehlen kann. Es ist durchaus möglich, die Praktikumszeit auf zwei Stellen aufzuteilen, jedoch wird von vielen Stellen eine Mindest-Dauer von vier Wochen gefordert, so dass du dann vielleicht 8 Wochen leisten musst. Das ist aber alles andere als schlimm, denn diese Praktika gelten als „Eingangstüren“ in einen Job. Also lieber mehr, denn weniger machen!

\subsection*{Wissenschaftliches Arbeiten und Bachelorarbeit}
Das Modul Wissenschaftliches Arbeiten soll als direkte Vorbereitung zur Bachelorarbeit dienen, wobei es von der Themenfindung über den Rechercheprozess bis zum eigentlichen Schreiben geht. Die Prüfungsordnung schreibt als Voraussetzung 100 Leistungspunkte vor, um mit der Bachelorarbeit zu beginnen. Weiterhin steht dort: „Die Studierenden sind in der Lage, eine konkrete Fragestellung aus dem Gebiet der Landschaftsökologie fachlich kompetent mit wissenschaftlichen Methoden selbständig und in vorgegebener Frist zu bearbeiten.“ Das Thema oder Themengebiet mit dem dazugehörigen Betreuer/Prüfer kann man sich frei aussuchen. Die Bearbeitungszeit beträgt ca. 9 Wochen und die Arbeit erbringt 12 Leistungspunkte.

\subsection*{Exkursionstage}
Neben Vorlesungsbesuchen und Geländepraktika gibt es im B.Sc. LÖK als weiteres Highlight die Exkursionstage. Es müssen insgesamt 8 Exkursionstage einer "großen" Exkursion (ca. 1 bis 2 Wochen) mit einem Begleitseminar nachgewiesen werden.
Alternativ kann man auch 12 Exkursionstage machen, die aus mehreren kleineren Exkursionen bestehen. Diese müssen nicht zwingend mehrtägig sein und haben meist kein Begleitseminar. Als Leistungsnachweis werden oft Referate, Exkursionsberichte und/oder Protokolle verlangt. Plätze sind begehrt, daher sollte man sich immer gut an der Exkursionspinnwand im Foyer und auf der Internetseite vom Front Office über Exkursionsangebote informieren. Die Anmeldung erfolgt in der Regel über den Exkursionspool des Front Office. Generell machen Exkursionen viel Spaß und das im Semester Gelernte kann praktisch angewendet werden!

\subsection*{Verwandte Studiengänge}
Ökologie-Studiengänge unterschiedlichster Art schießen wie Pilze aus dem Boden. Inzwischen haben sich einige Studiengänge etabliert. Neben Landschaftsökologie in Münster gibt oder gab es beispielsweise noch:
\begin{itemize}
  \item Landschaftsökologie (in Greifswald, Oldenburg)
  \item Geoökologie (in Bayreuth, Braunschweig, Freiberg, Karlsruhe und Potsdam)
  \item Ökologie (in Essen)
  \item Umweltwissenschaften (in Greifswald, Oldenburg ...) etc.
\end{itemize}
Zusätzlich gibt es weitere Studiengänge, in denen planerische und ökologische Zusammenhänge gelehrt werden, die denen der Landschaftsökologie in Münster oft ähnlich sind. 

\subsection*{Berufsfelder, der Blick in die Zukunft}
Für Landschaftsökologen gibt es kein einheitliches Berufsbild, wie zum Beispiel für Juristen, Chemiker oder Lehrer. Die Einsatzmöglichkeiten auf dem Arbeitsmarkt sind vielfältig, einige sind bei
\begin{itemize}
  \item (Landschafts-) Planungsbüros oder Beratungsunternehmen
  \item diversen Umweltbehörden auf verschiedenen Ebenen von der Gemeinde bis zur EU
  \item Einrichtungen des Natur- und Umweltschutzes, sowohl staatlich als auch privat, z.B. Biologische Stationen, Naturschutzverbände, Museen oder Naturschutzzentren
  \item Fachabteilungen oder -stellen in Unternehmen, in denen fächerübergreifende Kompetenz gefragt ist, z. B. bei Versicherungen, Industrie
  \item{Forschung}
\end{itemize}
Natürlich können sich auch Landschaftsplaner, Landschaftspfleger, Landschaftsarchitekten, Geoökologen, Geographen, Geophysikern, Biologen, Umwelttechnikern etc. auf mögliche Stellen bewerben. Doch kein Grund zum Verzweifeln: durch die Interdisziplinarität unseres Studiengangs ergeben sich vielfältige Möglichkeiten und „grüne“ Berufe sollen ja bekanntlich Zukunft haben! Also nicht verzagen, man findet schon seine ökologische Nische ;)


