\chapter{Einführung}
Herzlich Willkommen an den Instituten für Geographie, Didaktik der Geographie, Landschaftsökologie und Geoinformatik des Fachbereichs 14 Geowissenschaften der Westfälischen Wilhelms-Universität Münster. Mit diesem kleinen Info-Heft möchten wir euch einen kurzen Überblick über euer Studienfach geben. Dabei sei darauf verwiesen, dass wir uns bei der Erstellung an die jeweils gültigen Studien- und Prüfungsordnungen gehalten haben. Dabei haben wir versucht, eure Studiengänge etwas anschaulicher darzustellen. Für die Richtigkeit der Angaben bzw. Änderungen übernehmen wir keine Gewähr! Im Zweifelsfalle solltet ihr entweder direkt bei uns in der Fachschaft oder bei euren zuständigen FachstudienberaterInnen nachfragen.

\section*{Wichtiger Hinweis:}
\textbf{Dieses Info-Heft kann die zurzeit gültigen Prüfungs- und Studienordnungen nicht ersetzen, denn diese sind die rechtlichen Grundlagen eures und unseres Studiums.\newline Die hier dargestellten Angaben sind nicht rechtsverbindlich!!}

Wir raten euch, rechtzeitig mal einen Blick in eure Prüfungs- und Studienordnung zu werfen, damit ihr im Zweifelsfall wisst, wo ihr was nachlesen könnt. Eure Prüfungs- und Studienordnung findet ihr auf den Internetseiten eurer Institute.

\section*{Was machen wir?}
Wir vertreten die Interessen der Studierenden unseres Faches gegenüber einzelnen Dozierenden, den Instituten (Geographie / Landschaftsökologie / Didaktik der Geographie / Geoinformatik) oder dem Dekanat. Wir knüpfen und halten Kontakte zu Fachschaften anderer Hochschulen verwandter Fachrichtungen, um in dieser Hinsicht einen Informationsaustausch zu gewährleisten. Was ihr wahrscheinlich als erstes von uns mitbekommen werdet, sind unsere so genannten „Serviceleistungen“ wie die Orientierungswoche und das Erstsemester-Kennenlernwochenende in Drübberholz.
\\
Außerdem machen wir folgende tolle Sachen für euch:
\begin{itemize}
 \item Präsenzdienst im Fachschaftsraum
 \item Studienberatung
 \item Verleih von Klausuren, Prüfungsprotokollen, Skripten u.ä.
 \item Organisation der legendären Geopartys
 \item usw. usf. etc. pp.
\end{itemize}
Die Hartgesottenen von uns, die gar nichts abschreckt, setzen sich für euch an vorderster Front in Instituts- und Fachbereichsgremien ein
(siehe Kapitel „Gremien der Uni“), um dort eure (äh, auch unsere) Interessen durchzusetzen.
\begin{center}
....und bei so viel schönen Sachen,\\
lohnt es sich auch mitzumachen...
\end{center}
Und wer sich unseren bunten Haufen aus der Nähe anschauen möchte, kommt am besten einfach mal vorbei.
Wir treffen uns einmal pro Woche zur Fachschaftssitzung.\\
\textbf{In der Regel mittwochs ab 18:00 Uhr!}

\section*{Wahlen}
Ihr, also alle Studendierenden eines Fachbereichs der WWU (in eurem Fall des Fachbereichs Geowissenschaften), wählt ein Mal im Jahr die Mitglieder des Studierendenparlamentes und die Mitglieder der Fachschaftsvertretungen (welche wiederum den Fachschaftsrat wählen und letztere werden im Alltag „die Fachschaft“ genannt).

Als Vertreter gegenüber der Universitätsverwaltung und den Professoren brauchen wir natürlich einen gewissen Rückhalt von euch StudentenInnen. Deshalb geht wählen, wann immer möglich!!!

Am Rande (aber wichtig!): Bei den Wahlen ist die Wahlbeteiligung leider immer sehr gering (oft weit weniger als 20 \%)! Als gewählte Vertreter benötigen wir allerdings eure Unterstützung durch die Wahlbeteiligung. Eine Legitimation haben wir erst, wenn möglichst viele ihre Stimme abgeben. Wahlurnen sind niemals zu übersehen und beißen nicht, kosten nix und sind weder spießig noch prollig!!!

\section*{Studienbeitrag}
Seit dem Wintersemester 11/12 gibt es an dieser Universität keine Studiengebühren mehr.
Allerdings wird jedes Semester ein Semesterbeitrag erhoben, der zurzeit bei knapp 240 Euro liegt und für Dinge wie Studierendenwerk (Mensa, Wohnheime,.... ), Semesterticket und studentische Selbstverwaltung erhoben wird.
