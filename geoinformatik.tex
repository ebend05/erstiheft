\chapter{Bachelor of Science Geoinformatik}
\lohead{\footnotesize{Geographie - Landschaftsökologie - \textbf{Geoinformatik}}}
\rehead{\footnotesize{Geographie - Landschaftsökologie - \textbf{Geoinformatik}}}

\section{Allgemeines}

\emph{"`Fabian W. war mit 3 Freunden hier: Institute for Geoinformatics (ifgi)"'}.\\
So einen ähnlichen Satz wird wohl jeder schon einmal in einem sozialen Netzwerk gelesen haben. Und wenn man mal nicht weiß wie man zur WG-Party am Rudolf-Harbig-Weg kommt, hilft einem schnell das Smartphone. Auch der beste Weg zum Institut für Landschaftsöklogie ist mit Hilfe von GoogleMaps direkt gefunden. Und aus Autos sind Navigationsgeräte sowieso nicht mehr wegzudenken.\\
Meistens macht man sich schon gar keine Gedanken mehr, wie und warum diese Technologien funktionieren, da sie bereits ein fester Bestandteil unseres Lebens geworden sind. Doch irgendwo müssen die ganzen Informationen herkommen und irgendjemand muss diese Daten so aufbereiten, dass das Handy sie korrekt anzeigen kann, oder das Navi einen nicht an der nächsten Ecke in den Aasee fahren lässt.\\
Die Aufbereitung dieser Daten, die Sicherung der Datenqualität, und die Bereitstellung gehören zu eurem Aufgabengebiet. Und der Weg von einem Satellitenbild zu einem abstrakten dynamischen Graphen mit Straßeninformationen ist lang. Für die korrekte Einbindung aktueller Stau- und Wetterdaten sind sowohl Kenntnisse aus den Geowissenschaften als auch aus der Informatik notwendig. Und ein Routenplaner ist nur ein Beispiel aus der großen und ständig wachsenden Welt der Geoinformatik.\\
Geoinformatik ist anwendungsbezogen und kann als Schnittstelle zwischen etablierten geowissenschaftlichen Studiengängen (Geographie, Geologie, Landschaftsökologie, etc.) und den naturwissenschaftlich ausgerichteten Fächern (Mathematik, Informatik) betrachtet werden. Die Grenzen zwischen den Anwendungsgebieten verschwimmen oftmals und durch das interdisziplinäre Studium wird eine hohe Flexibilität erreicht, so dass ein ausgebildeter Geoinformatiker in einem breiten Aufgabenspektrum einsetzbar ist. Er verfügt über die benötigten Kenntnisse in den Geowissenschaften und der Informatik und so stehen ihm -- im Gegensatz zu hoch spezialisierten Abgängern anderer Studienfächer -- zahlreiche Möglichkeiten des beruflichen Werdegangs offen. Nicht ungern wird in unserem Institut immer wieder auf "`die 80 \%"' verwiesen.\\
Diese magische Zahl soll der Anteil der Fragestellungen sein, welche einen Raumbezug hat. Bei diesen Fragestellungen kann ein Geoinformatiker helfen eine bessere Antwort zu suchen. Typische Berufsfelder lassen sich in drei grobe Bereiche trennen. Behörden zeigen großes Interesse an jemandem, der bei Stadtplanung und mehr softwaretechnisch aushelfen kann. An den Universitäten wird in diesem jungen Forschungszweig immer nach neuen Wissenschaftlern gesucht. Und in der freien Wirtschaft werden die Anwendungen entwickelt, die woanders erfolgreich zum Einsatz kommen. Diese Bereiche sind aber auch wieder verzahnt, und für die möglichst reibungslose Kommunikation könnten und sollten an beiden Enden Geoinformatiker sitzen, denn nur wir verstehen sowohl die Fachsprache und Denkweise der Informatiker als auch die der Geowissenschaftler.

\section*{Geoinformatik studieren}
In Münster Geoinformatik zu studieren bedeutet weit mehr als Vorlesungen und Klausuren. Münster ist eine Stadt der Studentinnen und Studenten und es ist praktisch jeden Abend etwas los. Ist gerade mal keine Mathe- oder Geoparty, kann man sich in der Jüdefelder oder am Hawerkamp die Zeit vertreiben.

Aber zurück zum Studium – BSc. Geoinformatik: Der Bachelor in Geoinformatik hat natürlich das bereits beschriebene Grundgerüst aus Modulen, Zensuren, Creditpoints und einer abschließenden Bachelorarbeit. In den ersten beiden Semestern solltet ihr nach Plan möglichst den Grundstein in Mathematik und Informatik, sowie in den geoinformatischen Disziplinen, wie Geostatistik legen. Im zweiten Studienjahr erwarten euch die Grundbausteine in den Geowissenschaften. Die Kenntnisse in Informatik und Geoinformatik werden vertieft. Im letzten Studienjahr entscheidet ihr euch je nach Interesse für einige der angebotenen Seminare und arbeitet somit auf eure Bachelorarbeit und euren akademischen Abschluss hin.

Wenn ihr euch mit alten Freunden aus der Schule oder neuen Freunden an der Uni unterhaltet, die geisteswissenschaftliche Fächer studieren, werdet ihr schnell merken, dass Übungszettel charakteristisch für euer Studium sind. Das sind wöchentliche Hausaufgaben, die es mit 2 oder 3 Leuten zu bearbeiten gilt. Auch hierbei lernt man nette Leute kennen und kann weitere Kontakte knüpfen. Eure vorhin erwähnten Kommilitonen müssen dafür Unmengen an Hausarbeiten schreiben während ihr eure erste und einzige Hausarbeit im BSc. Geoinformatik im vierten Semester im Rahmen des Wahlfachs Humangeographie verfassen müsst. Diese ist auch die einzige methodisch-praktische Vorbereitung auf eure spätere Abschlussarbeit. Deswegen gibt es auch noch das "`General Studies/Allgemeinen Studien"'-Modul, dessen Kurse teilweise vorgegeben sind, aber auch selbst gestaltet werden können. So könnt ihr beispielsweise eine neue Sprache lernen oder eben einen Kurs wählen, der euch methodisch auf eure Abschlussarbeit vorbereitet. Die meisten Veranstaltungen gehen mit in die Endnote ein, das heißt, dass ihr möglichst alle Veranstaltungen gut abschließen solltet, aber andererseits fällt eine vergeigte Prüfung nicht so stark ins Gewicht.

Wenn alle Module geschafft sind und die Bachelorarbeit geschrieben ist, hast du die Möglichkeit auch noch den Master of Science Geoinformatics zu absolvieren und so eine noch höhere Qualifikation zu erlangen. Dazu sind einige Zugangsvoraussetzungen, wie ein Englischtest, erforderlich (Der Master-Studiengang wird komplett auf englisch bestritten). Dazu und zu allen anderen Fragen bzgl. des Masters wirst du aber in deinen drei Jahren Bachelorstudium die passenden Antworten bekommen. Ihr könnt euch jederzeit gerne telefonisch oder per Mail (davon wirst du eine ganze Menge in deinem Studium schreiben) an uns wenden. Am besten ist es aber immer, wenn ihr persönlich bei uns vorbei kommt. Wo ihr uns findet ist im Abschnitt "`Fachschaft"' beschrieben. Wenn wir euch einmal nicht weiterhelfen können, könnt ihr euch gerne an unsere Bachelorstudienberater Prof. Dr. Angela Schwering und Thomas Bartoschek wenden.

Im Folgenden sind alle Module aufgelistet, die ihr in eurem Studium zu absolvieren habt. Diese könnt ihr ebenfalls auf der ifgi-Webseite (\url{http://www.bachelor-geoinformatik.de/}) im Bereich "`Prüfungen"' finden.

\newpage

\section{Modulübersicht}

%Dies ist die vorläufige Modulübersicht des Bsc. Geoinformatik. Uns lag zum Zeitpunkt des Drucktermins noch nicht die finale Version vor, es sollte sich allerdings nicht mehr viel ändern. Bei Fragen wendet euch direkt an uns!

\begin{longtable}{p{0.7\textwidth} p{0.12\textwidth} p{0.1\textwidth}}
 & Typ & ECTS \\
\textbf{Geoinformatik 1 -- Grundlagen} & & \textbf{5}\\
Einführung Geoinformatik & V & 3\\
Einführung Geoinformatik & Ü & 2\\
& &\\
\textbf{Geoinformatik 2 -- Angewandte Kartographie} & & \textbf{7}\\
GIS Grundkurs & Ü & 2\\
Angewandte Kartographie & Ü & 5\\
& &\\
\textbf{Geoinformatik 3 -- Geostatistik} & & \textbf{5}\\
Einführung in die Geostatistik & V & 2\\
Einführung in die Geostatistik & Ü & 3\\
&&\\
\textbf{Geoinformatik 4 -- Dynamische räumliche Prozesse} & &\textbf{5}\\
Einführung in die Modellierung dynamischer räumlicher Prozesse & V & 2 \\
Einführung in die Modellierung dynamischer räumlicher Prozesse & Ü & 3\\
&&\\
\textbf{Geoinformatik 5 -- Fernerkundung} && \textbf{5}\\
Einführung in die Fernerkundung & V & 2\\
Einführung in die Fernerkundung & Ü & 3\\
&&\\
\textbf{Geoinformatik 6 -- Interoperabilität}&& \textbf{10}\\
Geodateninfrastrukturen und Geoinformationsdienste (SII) & V & 2\\
Geodateninfrastrukturen und Geoinformationsdienste (SII) & Ü & 3\\
Reference Systems for Geoinformation & V & 2\\
Reference Systems for Geoinformation & V & 3\\
&&\\
\textbf{Geoinformatik 7 -- Softwareentwicklung}&& \textbf{15}\\
Geosoftware I & P & 6\\
Geosoftware II & P & 9\\
&&\\
\textbf{Geoinformatik 8 -- Perspektiven}&& \textbf{8}\\
Geoinformatik Seminar & S & 3\\
Ausgewählte Probleme der Geoinformatik (Wahlpflicht) & V/Ü/S & 5\\
&&\\
\textbf{Mathematik}&& \textbf{20}\\
Analysis für Informatiker  & V\,+\,Ü & 10\\
Lineare Algebra für Informatiker & V\,+\,Ü & 10\\
&&\\
\textbf{Informatik 1: Grundlagen der Programmierung} & & \textbf{12}\\
Informatik 1 & V\,+\,Ü & 5+4\\
Java Programmierkurs & V\,+\,Ü & 3\\
&&\\
\textbf{Informatik 2: Algorithmen und Datenstrukturen} & & \textbf{12}\\
Informatik 2 & V\,+\,Ü & 5+4\\
C/C++ Programmierkurs & V\,+\,Ü & 3\\
&&\\
\textbf{Informatik 2: Datenbanken} & & \textbf{7}\\
Datenbanken & V\,+\,Ü & 4+3\\
&&\\
\textbf{Informatik 4: Software-Entwicklung}& &\textbf{6}\\
Software-Entwicklung & V\,+\,Ü & 4+2\\
&&\\
\textbf{Informatik 5: Vertiefung} & & \textbf{6}\\
Diskrete Strukturen (Wahlpflicht) & V\,+\,Ü & 6\\
Computergrafik (Wahlpflicht) & V\,+\,Ü & 6\\
Computer Vision (Wahlpflicht) & V\,+\,Ü & 6\\
Algorithmische Geometrie (Wahlpflicht) & V\,+\,Ü & 6\\
&&\\
\textbf{Geowissenschaften 1: Physische Geographie} & & \textbf{10}\\
Physische Geographie & V & 5\\
Geländeübung Physische Geographie & Ü & 5\\
&&\\
\textbf{Geowissenschaften 2a: Humangeographie} & & \textbf{10}\\
Einführung Humangeographie & V & 5\\
Eine Übung zur Humangeographie & Ü & 4\\
Exkursion & & 1\\
&&\\
\textbf{Geowissenschaften 2b: Orts-, Regional- und Landesplanung} & & \textbf{10}\\
Grundlagen der Raumplanung & V & 3\\
Einführung in die räumliche Planung & S & 6\\
Exkursion & & 1\\
&&\\
\textbf{Geowissenschaften 3a: Vertiefung Geologie} & & \textbf{5}\\
Die Erde & V & 5\\
&&\\
\textbf{Geowissenschaften 3b: Vertiefung LÖK} & & \textbf{5}\\
Klimatologie\,/\,Hydrologie\,/\,Vegetations- oder Tierökologie& V\,+\,Ü & 5\\
&&\\
\textbf{General Studies} & & \textbf{18}\\
Präsentation, Rhetorik, Fremdsprachen & V/Ü/S/P & 8\\
Projektplanung und Projektmanagement & Ü & 5\\
Projekt & P & 5\\
&&\\
\textbf{Thesis} & & \textbf{14}\\
Bachelor-Abschlussarbeit & & 12\\
Blockkurs Vorbereitung Bachelor-Abschlussarbeit & S & 2\\

\end{longtable}

Alle Angaben ohne Gewähr. (Vergleiche \url{http://www.bachelor- geoinformatik.de/inhalte-des-studiums/pruefungsordnung})

\newpage

\section*{Fachschaft}
Seit dem Wintersemester 2001/02 hat unser Studiengang eine eigene Fachschaft, die sich um die speziellen Fragen und Wünsche der Geoinformatikstudierenden kümmert. Die Fachschaft besteht aus gewählten Vertretern der Studenten eines Faches. In der Fachschaft versuchen wir die Interessen der Studierenden gegenüber der Hochschule zu vertreten und Konflikte und Unklarheiten zu beseitigen. Also sind wir eine Anlaufstelle für Studierende und Dozenten gleichermaßen, die die Kommunikation unter den Studierenden und mit den Dozenten fördert, z.B. durch übergreifende Veranstaltungen. Die Fachschaft will die Interessen der Studierenden nach bestem Wissen und Gewissen kundtun, vertreten und verteidigen. Dies gilt natürlich besonders für euch Studienanfänger. Dieses Heft zum Beispiel wurde von Fachschaftsmitgliedern entworfen und wir hoffen, euch damit einen ersten Überblick über einen innovativen und jungen Studiengang gegeben und vielleicht sogar euer Interesse am Mitwirken in der Fachschaft geweckt zu haben. Abschließend möchten wir euch auf diese beiden Homepages aufmerksam machen:

\begin{center}
\textbf{\url{ifgi.uni-muenster.de}}\\
\end{center}

ifgi steht für "`Institut für Geoinformatik"'. Hier findet ihr alle Informationen zum Studiengang wie die Studienordnung, eine Übersicht über die Module oder ein einfaches FAQ. Die Website bietet ausreichend Informationen bezüglich aller Kurse und Vorlesungen die am ifgi angeboten werden.

\begin{center}
\textbf{\url{geofs.uni-muenster.de}}\\
\end{center}

Die Homepage der Fachschaft. Sie wird von den Fachschaftlern ständig aktualisiert und ergänzt. Wir bemühen uns, den Studierenden, gerade Studienanfängern, mit dieser Homepage das Studium zu erleichtern. Also schaut euch die Seite an und zögert bei Fragen nicht, eine E-Mail an einen Vertreter oder die Fachschaft direkt zu schicken. Uns liegt viel daran, euch zu helfen und eventuelle Fragen oder Probleme schon früh zu klären.

Zum Schluss wünschen wir euch ein erfolgreiches und unterhaltsames Studium!

\newpage

\section*{FAQ} % alles in bf sind die fragen
\textbf{Buäh, jetzt brummt mir aber der Kopf. Die ganzen Infos muss ich erst einmal verdauen. Wenn ich noch andere Fragen bzgl. des Studiums habe, wo seid ihr zu finden?}

Wir als Fachschaft sind direkt gegenüber des Hörsaals im Erdgeschoss des GEO1 an der Heisenbergstraße zu finden. Bitte erkundigt euch vorher auf unserer Homepage nach unseren Präsenzzeiten!\\

\textbf{Wie stark unterscheidet sich Geoinformatik von der normalen Informatik? Ich habe irgendwie die Hoffnung, dass das alles etwas praxisorientierter und nicht ganz so bieder und trocken ist...}

Es ist richtig, dass Informatik teilweise bieder und trocken ist. Es ist auch richtig, dass Mathematik ziemlich langweilig sein kann. Es ist aber auch richtig, dass ORL (Orts-, Regional- und Landesplanung) und Gesteinskunde trockene Lernfächer sind. Abgesehen davon ist das alles subjektiv. Die ersten Semester der Kerninformatik sind denen der Geoinformatik (sowie der Wirtschaftsinformatik) recht ähnlich - und immer schwierig. Die Spezialisierung findet später statt, und ist auch bei den Informatikern sicherlich spannend. Wenn man Mathematik immer langweilig fand und mit Computern nichts am Hut hat, sollte man sich Geoinformatik aus dem Kopf schlagen. Ansonsten lässt sich mit einer gehörigen Portion Sturheit jede Informatikprüfung schaffen.\\

\textbf{Wie sieht's mit Mathevorkenntnissen aus? Ich habe nur Mathematik Grundkurs gehabt. Reicht das aus?}

Es kommt vor allem auf den Willen an, sich mit dem Stoff auseinanderzusetzen. Das Schulwissen ist mit den Mathematikveranstaltungen an der Universität nicht vergleichbar; das komplette Gebiet wird noch mal vollständig aufgerollt. Der frühere Unterschied zwischen Grundkurs und Leistungskurs ist fast gar nicht mehr vorhanden. Die Mathematik-Vorlesungen sind dennoch sehr anspruchsvoll und zeitintensiv, weshalb ihr hier mit völliger Mathematikphobie fehl am Platz seid. Bei uns hatten bei weitem nicht alle einen Mathe-LK. Einige waren sogar immer richtig schlecht in Mathe ;-)\\

\textbf{Gibt es auch eine Erstifahrt?}

Ja. Natürlich. Wo denkst du hin?\\

\textbf{Lohnt sich das Ersti-Wochenende? Was passiert da so im Allgemeinen?}

Klar. Man rennt die ganze Zeit im Wald rum, lernt seine Mitstudis kennen und lieben und hat jede Menge Spaß dabei. Wer nicht mitfährt ist selbst Schuld. Wer mehr wissen will, fragt einfach mal jemanden aus der Fachschaft oder schaut auf der Homepage nach.\\

%\textbf{Wo oder was ist StudLab A?}

%Ein StudLab ist ein großer Raum mit ziemlich vielen, manchmal funktionierenden Rechnern. Wir Geoinformatiker haben aber die dumme Angewohnheit, die meiste Zeit vor diesen grauen Kisten zu sitzen. Weil man im ersten Semester normalerweise noch keinen Laptop mit dem kompletten Set an Geoinformatik-Software im Regal liegen hat, trifft man sich in den StudLabs (früher wurden diese CIP-Pools genannt). Da lernt man dann auch, ICQ-Nicks Gesichtern zuzuordnen.

%Das StudLab A findet sich im ersten Stock des Instituts für Landschaftsökologie an der Robert-Koch-Straße 28, direkt neben dem Raum der Fachschaft GeoLök.

%Ausserdem gibt es noch die StudLabs B und C, diese befinden sich in zwei Containern am Ende der Robert-Koch-Str. Im ifgi gibt es im 5. Stock ebenfalls ein StudLab.\\

\textbf{Hat Geoinformatik was mit Geologie zu tun?}

Wenn man möchte. Man kann zwischen verschiedenen Themen der Landschaftsökologie und einer Vorlesung in der Gesteinskunde wählen. Die Geoinformatik hat aber nichts mit dem Berufsbild eines Geologen zu tun. Der Geologe geht raus ins Gelände und schaut, ob das Grundstück passend für ein 80-Stockwerke-Hochhaus ist (nachdem er auf die Altlasten hingewiesen hat). Ein Geoinformatiker sitzt dann später irgendwo in diesem Hochhaus. Und beide werden sich nicht mehr an die Geologievorlesung erinnern können. Außer daran, dass sie wirklich spannend war. Zu der soliden Grundbildung eines Geoinformatikers gehört sicherlich auch die Geologie. Aber genauso wichtig sind die Geographie- und Landschaftsökologieveranstaltungen.\\

\textbf{Ich hab da was gehört, wenn ich dreimal durch eine Klausur falle, bin ich exmatrikuliert, stimmt das?}

Grundsätzlich gibt es Studien und Prüfungsleistungen in eurem Studium. Klausuren die als Studienleistung zählen, können beliebig oft wiederholt werden. Tatsächlich ist diese 3-mal-Durchfallregelung (schönes Wort) mit anschließender Exmatrikulation nur bei Prüfungsleistungen der Fall. Grundsätzlich gilt: Klausuren in Kursen, welche nicht in die Modulabschlussnote einfliessen, dürfen beliebig oft wiederholt werden (Studienleistungen). Dennoch sollte man dies nicht als Freifahrtschein ansehen, da gerade diese Kurse einen wichtigen Grundstein für das weitere Studium legen.
Sollte tatsächlich der Fall eintreten, dass ihr eine wichtige Klausur zum dritten Mal nicht bestanden habt, tritt tatsächlich der Fall ein, dass ihr exmatrikuliert werdet. Eine Besonderheit ist z.B. das Modul Geowissenschaften 2, in dem nach zwei Fehlversuchen in einem Fach auf das andere Wahlpflichtfach gewechselt werden kann, ohne seine zwei Patzer mitnehmen zu müssen.\\

\textbf{Beißen Dozenten?}

Nein, zumindest wurde noch keiner dabei beobachtet. Scheut euch also nicht auch mal nach einer Veranstaltung mit wichtigen Fragen zu eurem Dozenten zu gehen oder euch mal in eine Sprechstunde zu setzen.

Des Weiteren gibt es auch auf der offiziellen ifgi-Webseite\footnote{http://ifgi.uni-muenster.de} eine solche Fragensammlung. Für weitere Fragen solltet ihr einfach mal in der Fachschaft vorbeischauen. Die aktuellen Präsenzzeiten findet ihr immer auf unserer Webseite\footnote{https://geofs.uni-muenster.de}.

%\begin{center}
%\includegraphics[scale=0.3]{cartoonC}
%\end{center}
