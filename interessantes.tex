\chapter{Interessantes} 

\section{Erasmus-Erfahrungsbericht Salamanca 2012/2013}

Salamanca. Eine mittelgroße Stadt in der Hochebene Kastiliens, drum herum nur Einöde, kaum ein Strauch. Im Sommer Hitze, im Winter gar nicht mal so warm. Was bringt einen dazu hier Erasmus zu machen, grade wenn auch Ziele wie Las Palmas de Gran Canaria zur Verfügung stehen? 

Die Antwort will ich versuchen im folgenden Bericht zu geben, doch erst einmal mal wieder alles auf Anfang. Die Planung und Bewerbung lief unkompliziert, bedingt durch die gute Hilfe durch das Erasmus Büro und auch keine besonderen Probleme bei der Uni Salamanca. Da dies schnell unter Dach und Fach war, hieß es nur noch: Flug buchen, (von Weeze nach Madrid), Bus buchen (direkt vom Terminal 1 des Flughafen Madrids mit avanzabus.com nach Salamanca) und sich in einem der zahlreiche Hostels einbuchen. Zwei bis drei Nächte sind dabei ausreichend, denn die Wohnungssuche gestaltet sich dank großem Angebot sehr einfach. Tipps dazu gibt es auch (zu mindestens auf Nachfrage) beim Check-in im Erasmus-Büro der Uni Salamanca. Man sucht am besten nach einer WG- Unterschied zu Deutschland: Man lernt die Mitbewohner meist nicht beim Casting kennen, sondern der Vermieter übernimmt die Auswahl. Auf Nachfrage kann man aber zu mindestens rausfinden ob die Mitbewohner männlich oder weiblich sind, spanisch sind oder auch Erasmus machen. Mit der Wohnung in der Tasche, kann man sich auf die Wahl seiner Kurse konzentrieren- denn eingerichtet sind die Wohnungen schon! Man sollte auf eine gute Küchenausstattung achten, denn die Mensa ist recht teuer. Auch Bettzeug sollte man besser von Zuhause mitbringen, Matratze ist aber meistens vorhanden. 
Zur Auswahl der Kurse: Man hat zwei Wochen Zeit sich alle Kurse anzugucken. Dabei sollte man vor allem auf eine gute Verständlichkeit der Profs achten, denn inhaltlich wird man sowieso nicht an das gewohnte deutsche Niveau herankommen- was aber nicht bedeutet das man weniger Arbeit hat: Hausarbeiten, Exkursionsberichte, Hausaufgaben und dann noch Klausuren werden meist gleichzeitig gefordert. Auch werden in einigen Kursen Exkursionen angeboten, so zum Beispiel im Kurs „Geografia de Turismo“, was natürlich ideal zum Entdecken des Landes ist. Der Kurs ist zwar inhaltlich nicht überragend, aber durch die Exkursionen und die Hausarbeit doch recht empfehlenswert. Nicht wundern sollte man sich wenn nur Erasmusstudenten im Kurs sitzen: Die ist bei Wahlpflichtfächern öfters der Fall- und bringt auch Vorteile, da allem „im selben Boot“ sitzen- und natürlich trotzdem Spanisch gesprochen wird. Zum Einschreiben sollte man genügend Passfotos bereithalten (pro Kurs eins), wobei diese auch einfach auf Papier ausgedruckt werden können 
Der Alltag gestaltet sich einfach in Spanien. Die anfangs noch unsicheren Spanischkenntnisse werden schnell zu mindestens in Alltagssituationen sicher und Supermärkte und Fruterias gibt es an jeder Ecke. Nur die Siesta lässt einen zu Anfang schimpfen. die Läden machen am frühen Nachmittag für 2-3h dicht. Fahrräder können von der Uni für relativ wenig Geld ausgeliehen werden, aber innerhalb der Stadt läuft man meist zu Fuß. 
Ausflüge kann man vor allem in die vielen historischen Städte Kastiliens mit dem Bus machen, auch ein Ausflug mit Mietwagen in andere Städte Spaniens oder Portugals bietet sich an. Wer in die Natur will, wird in der direkten Umgebung enttäuscht, dafür bieten sich aber Exkursionen der Uni an. Die Berggruppe der Uni bietet 1-2 mal im Monat Wanderexkursionen zu je 16€ an, dabei sind Wanderschuhe Pflicht! Die Exkursionen lohnen sich wirklich gut um das Umland und vor allem die 
Berge kennenzulernen, auch trifft man nette Leute. Die Anmeldung läuft über das Sportbüro der Uni, wo man auch Informationen über das weitere Sportangebot findet. Ich habe zum Beispiel noch einen Pádel-Kurs gemacht, eine Mischung zwischen Tennis und Squash. Es ist wirklich ein sehr spaßiges Spiel, nur leider kann ich den Sport in Deutschland nicht fortführen, er ist ein typisch spanisches Phänomen. Schläger muss man sich übrigens selbst besorgen (gilt auch für den Tenniskurs), es gibt aber einen Decathlon (sehr günstiger Sport-Großmarkt) in der Stadt. Für Schwimmer: Wer die städtischen Bäder nutzen will sollte beachten, dass Badekappe, Schwimmbrille und Badelatschen Pflicht sind, also gleich mitbringen. 
Nun will ich abschließend versuchen nochmal direkt auf die anfangs gestellte Frage antworten: Nein, Salamanca ist kein Überwinterungsparadies, es regnet auch mal eine Woche oder es friert bei Nacht- aber dann hat es Anfang Januar auch mal 20 Grad. Nein, in Salamanca kann man nicht surfen oder eben mal aus der Stadt in den Wald oder in die Berge fahren, drum herum ist wirklich nichts, und es ist auch keine Großstadt sondern man kann die Stadt in zwei Tagen sehen und in einem Monat kennenlernen. 
Ja, Salamanca hat eine gewisse Anziehungskraft, durch sein historische Uni und sehr schöne Altstadt, durch sein Katalanisch und durch das muntere Studenten und Erasmusleben, grade bei Nacht. Und man kann unglaublich gut Tapas essen gehen, eine meiner Lieblingsbeschäftigungen. Alles ist zu Fuß zu erreichen und man hat kein Problem die neugewonnen Freunde zu besuchen, man trifft sich sogar immer wieder zufällig auf der Straße. Die Uni ist nicht schlecht organisiert, aber das spanische wissenschaftliche Niveau im Allgemeinen ist zu mindestens in der Geografie echt nicht gut. 
Diese negativen und positiven Seiten hat und am Ende auch der Grund war um zu sagen: Es war schön, aber ein halbes Jahr hat (für mich) gereicht- auch zum Spanisch lernen. 
 \\
\\
Für Fragen rund um das ERASMUS/SOKRATES Programm steht euch das Erasmus-Büro zu Verfügung. Hier bekommt ihr individuelle Beratung und alles Wissenswerte zum Bewerbungsverfahren.\\ \\
\textbf{Lisa Kamphaus, Katja Plumbaum, Simon Radtke}\\
\textbf{Erasmus-Büro}   \url{erasmus.geo@uni-muenster.de}	Tel.:\,83--33988\\ 

\section{Geld zu verschenken – Stipendienmöglichkeiten für Studis}
Miete, Internet und Handyrechnungen, außerdem Bücher, Exkursionen, Kaffe, Tiefkühlpizza – im Laufe eures Studiums kommen so einige Ausgaben auf euch zu. Wenn dann noch das eine oder andere ,Erfrischungsgetränk' zu sich genommen wird, herrscht auch schnell mal Ebbe im Portemonnaie. Um den Tidenhub wieder in vernünftige Bahnen zu bringen, gibt es ziemlich viele Möglichkeiten: Mama und Papa anpumpen, jobben gehen, Bafög beantragen, etc. Eine ergänzende Alternative ist ein Stipendium. Das klingt soweit super, aber was ist das, wer fördert eigentlich und wie kommt man da ran?
Stipendien können sehr verschieden aussehen. Manche Stiftungen unterstützten ‚nur’ finanziell, andere legen großen Wert auf Teilnahme an Seminaren oder Workshops. Auch die Bemessungsgrundlage der finanziellen Förderung variiert stark. Meist wird ein Betrag von \unit[60-100]{\officialeuro} pro Monat als so genanntes Büchergeld unabhängig von der finanziellen Situation des Stipendiaten ausgezahlt, eine etwaige weitergehende Förderung orientiert sich oftmals an den Bafög-Kriterien.
Die bekanntesten Stipendiengeber sind sicherlich die zwölf ‚großen’ Begabtenförderungswerke:
\begin{itemize}
 \item Cusanuswerk
 \item Deutschland-Stipendium
 \item Evangelisches Studienwerk e. V. Villigst
 \item Friedrich-Ebert-Stiftung
 \item Friedrich-Naumann-Stiftung
 \item Hans-Seidel-Stiftung
 \item Hans-Böckler-Stiftung
 \item Heinrich-Böll-Stiftung
 \item Konrad-Adenauer-Stiftung
 \item Rosa-Luxemburg-Stiftung
 \item Stiftung der Deutschen Wirtschaft
 \item Studienstiftung des Deutschen Volkes 
\end{itemize}
Diese vergeben die größte Anzahl Stipendien, verzeichnen allerdings auch die höchsten Bewerberzahlen. Daher ist häufig eine Bewerbung bei einer der unzähligen kleineren Stiftungen aussichtsreicher. Einige der Stiftungen haben aufgrund ihres geringen Bekanntheitsgrades sogar so wenige Bewerbungen, dass nicht alle Stiftungsmittel ausgezahlt werden. Zum Teil sind die Förderungen regional begrenzt oder nur für bestimmte Fachrichtungen offen, daher kann es eine ganze Weile dauern bis man eine passende Stiftung gefunden hat. Um die Suche ein wenig abzukürzen, hat e-fellows.net, selbst ein Stipendiengeber, eine Datenbank erstellt, in der viele Stiftungen mit einer Kurzbeschreibung verzeichnet sind. Das ganze finde ihr unter folgendem Link:\\ 
\\
\url{http://www.e-fellows.net/show/detail.php/5789}\\
\\
Die genauen Anforderungen, Bewerbungsabläufe und Fristen sind sehr unterschiedlich und es kann eine ganze Weile dauern, bis man alle erforderlichen Unterlagen beieinander hat. Zudem ist eine Bewerbung teilweise nur bis zum 2. oder 3. Fachsemester möglich. Man sollte daher relativ frühzeitig mit der Suche beginnen, um nicht irgendwelche Fristen zu verpassen. Ganz wichtig ist auch, sich nicht abschrecken zu lassen durch die Anforderungen der Stiftungen oder den Aufwand, der mit einer Bewerbung verbunden ist. Sollte am Ende eine Zusage dabei herauskommen, hat sich der Aufwand in jedem Fall gelohnt und immer dran denken: Die Anderen kochen auch nur mit Wasser!

\section{Die Hilfskraft – Studenticus helpissimus}
Wer kennt sie nicht, die ominöse „Hilfskraft“?!? Sie begegnet einem in Seminaren, im Vorzimmer des Professors, an der Kaffeemaschine, in der Bibliothek oder am Kopierer. Aber was macht so eine Hilfskraft eigentlich wirklich und wieso steht sie dann doch manchmal an der Kaffeemaschine anstatt schlaue Bücher mit zu verfassen? Also in erster Linie geht es aus unserer Sicht darum, ein bisschen Geld zu verdienen. Je nachdem, wie viele Stunden pro Woche ableistet werden, kann eine SH (Studentische Hilfskraft) bis zu 450 Euro im Monat dazu verdienen. Das hört sich gut an, aber der Weg zum Geld kann doch recht steinig sein, vor allem sollte man nicht unterschätzen, dass man diese Zeit bei Frei- und Studienzeit abziehen muss. Außer dem Geld bekommt man aber auch spannende Einblicke in den Lehr- und Forschungsbetrieb an der Uni, je nachdem, um was für eine Stelle man sich bemüht hat.\par
Hilfskräfte gibt es an verschiedenen Orten: in den Arbeitsgruppen der Professorinnen und Professoren (Kopieren, Recherchieren, Korrektur"=Lesen, Schreibarbeit, etc.), in der Bibliothek oder im ZDM (Aufsicht, Hilfe, …) usw. Es empfiehlt sich zumeist, bereits ein bisschen Uni-Luft geschnuppert zu haben, also vielleicht nicht gleich im ersten Studienjahr anzufangen. Die Verfügbarkeit offener Stellen ist sehr unterschiedlich. Während beispielsweise die Geoinformatiker recht viele freie Hilfskraftstellen anbieten können, die dann über die Monitore flackern oder am Schwarzen Brett rumhängen, so muss man sich in den meisten Fällen selbst darum kümmern, indem man einfach nachfragt und seine Arbeitskraft anbietet. Am besten dort, wo einen das Thema oder die Arbeit auch ein bisschen interessiert. Und hier lautet die Devise, nicht enttäuscht zu sein, wenn es nicht sofort klappt, aber vielleicht kommt die- oder derjenige ja darauf zurück. Also erstmal eine Anfrage starten und Namen und Adresse da lassen, häufig wird dann später doch was draus. 
\par
Wie man letztendlich eingesetzt wird, hängt dann ganz vom Chef und den anstehenden Arbeiten ab. Tendenziell wachsen die Aufgaben mit dem fachlichen Wissen der Hilfskraft, was aber nicht bedeutet, dass jede Hilfskraft als Tellerwäscher anfängt. Zurzeit bekommt man neun Euro pro Stunde, wobei hier Vorsicht geboten ist, denn bei so manchen Chefs ist „eine Stunde Verdienst“ schnell auch mal zu mehr als zwei Stunden Arbeit geworden. Ob man das dauerhaft mit sich machen lässt, ist einem selbst überlassen. Zumeist sind die Arbeitsbedingungen allerdings wirklich fair und es herrscht eine gute Stimmung. Man darf den Kaffee also auch mittrinken, was die Motivation ihn zu kochen doch deutlich steigert, oder? \par
Letztendlich sind Hilfskraftstellen für all diejenigen zu empfehlen, die gerne etwas tiefer in die Forschungsarbeit oder die Lehre mit einsteigen wollen und die Uni nicht nur als Lernanstalt sehen. Zugegebenermaßen sind nicht alle Stellen anspruchsvoll, aber zum Geldverdienen taugen sie schon. Und wenn man feststellt, man mag die Arbeit, ergibt sich daraus vielleicht auch nach dem Studium noch eine berufliche Option, denn viele Doktoranden haben sich in ihrer Studienzeit bereits als Hilfskraft verdient gemacht. Sobald man übrigens einen fertigen Abschluss hat, kann man bereits als WH (Wissenschaftliche Hilfskraft) eingestellt. Das gibt dann einige Euro mehr pro Stunde. 

\newpage

\section{Ein Tag meines Lebens als Student}
(unbekannter Autor, aber war nach kleiner Bearbeitung sehr passend und durfte deshalb nicht fehlen)\\
\begin{verse}
\textbf{1. Semester}\\ 
%\\
05:30 Der Quarz-Uhr-Timer mit Digitalanzeige gibt ein zaghaftes "`Piep-Piep"' von sich. Bevor sich dieses zu energischem Gezwitscher entwickelt, sofort ausgemacht \linebreak und aus dem Bett gehüpft. Um die Promenade gejoggt, mit einem Besoffenen zusammengestoßen, anschließend eiskalt geduscht. \\
%\\
06:00 Beim Frühstück Greenpeace Magazin auswendig gelernt und Umweltpolitik der Grünen analysiert. Danach kritischer Blick in den Spiegel, Outfit genehmigt.\\ 
%\\
07:00 Zur Uni gehetzt. Hörsaal erreicht. Pech gehabt: erste Reihe schon besetzt. Niederschmetternd. Beschlossen, morgen doch noch eher aufzustehen.\\ 
%\\
07:30 Vorlesung. Keine Disziplin! Einige Kommilitonen lesen Sportteil der Zeitung oder gehen zum Frühstücken. Alles mitgeschrieben; Füller leer, aber über die Witzchen des Dozenten mitgelacht.\\ 
%\\
08:00 Vorlesung. Verdammt! Extra neongrünen Pulli angezogen und trotz eifrigem Fingerschnippens nicht drangekommen.\\
%\\
10:45 Nächste Vorlesung. Nachbar verlässt mit Bemerkung "`Sinnlose Veranstaltung"' den Raum. Habe mich für ihn beim Prof. entschuldigt.\\ 
%\\
12:00 Mensa Stammessen II. Nur unter größten Schwierigkeiten weitergearbeitet, da in der Mensa zu laut.\\ 
%\\
12:45 In Fachschaft gewesen. Skript immer noch nicht fertig. Wollte mich beim Vorgesetzten beschweren. Keinen Termin bekommen. Daran geht die Welt zugrunde.\\ 
%\\
13:00 Fünf Leute aus meiner 0-Gruppe getroffen. Gleich für drei AG's zur Klausurvorbereitung verabredet.\\
%\\
13:30 Dreiviertelstunde im Copyshop gewesen und die Klausuren der letzten 10 Jahre mit Lösungen kopiert. Dann Tutorium: Ältere Semester haben keine Ahnung.\\ 
%\\
15:30 In der Bibliothek mit den anderen gewesen. Durfte aber statt der dringend benötigen 18 Bücher nur vier mitnehmen.\\ 
%\\
16:00 Proseminar. War gut vorbereitet. Hinterher den Assi über seine Irrtümer aufgeklärt.\\ 
%\\
18:30 Anhand einschlägiger Quellen die Promotionsbedingungen eingesehen und erste Kontakte geknüpft.\\ 
%\\
19:45 Abendessen. Verabredung im "`Blauen Haus"' abgesagt. Dafür Vorlesungen der letzten paar Tage nachgearbeitet.\\ 
%\\
23:00 Videoaufzeichnung von Terra Nova angesehen und im Bett noch den Campbell gelesen. Festgestellt, 18\textminus{}Stunden\textminus{}Tag zu kurz. Werde demnächst die Nacht hinzunehmen.                                                                                                                                                                      \end{verse}

\newpage

\begin{verse}
\textbf{ 13. Semester }\\
10.30 Aufgewacht! Kopfschmerz. Übelkeit. Zu deutsch: KATER.\\ 
%\\
10.45 Der linke große Zeh wird Freiwilliger bei der Zimmertemperaturprüfung. (arrgh!) Zeh zurück. Rechts Wand, links kalt; Ich bin gefangen.\\ 
%\\
11.00 Kampf gegen den inneren Schweinehund: Aufstehen oder nicht -- das ist hier die Frage.\\ 
%\\
11.30 Schweinehund schwer angeschlagen, wende Verzögerungstaktik an und schalte Fernseher ein\\ 
%(inzwischen auch schon verkabelt).\\ 
%\\
12.05 Mittagsmagazin beginnt. Originalton Moderator: "Guten Tag liebe Zuschauer Guten Morgen liebe Studenten." - Auf die Provokation hereingefallen und aufgestanden.\\ 
%\\
13.30 In der Cafeteria der Mensa am Aasee beim Skat mein Mittagessen verspielt.\\ 
%\\
14.30 Im Gasolin hereingeschaut. Direkt Geld gepumpt und 'ne Kleinigkeit gegessen: Das Bier schmeckt auch wieder! Kurze Diskussion mit ein paar Leuten über die letzte Entwicklung des Dollar-Kurses und der Weltpolitik.\\ 
%\\
15.45 Kurz in der Bibliothek gewesen. Nur weg hier, total von Erstsemestern überfüllt.\\ 
%\\
16.00 Fünf Minuten im Tech gewesen. Nichts los! Keine Zeitung, keine Flugblätter - nichts wie raus.\\ 
%\\
%17.00 Stammkneipe hat immer noch nicht geöffnet.\\ 
%\\
18.15 Wichtiger Termin zuhause: Star Trek!\\ 
%\\
18:20 Mist! Kein Star Trek! Stattdessen Live-Übertragung von Stöhn-Seles. SAT 1 war auch schon besser...\\ 
%\\
19.10 Komme zu spät zum Date mit der hübschen blonden Erstsemesterin im Barzillus. Immer dieser Stress!\\ 
%\\
01.00 Die Kneipen schließen auch schon immer früher... Umzug in's Amp.\\ 
%\\
04.20 Tagespensum erfüllt. Das Bett lockt.\\ 
%\\
05.35 Auf der Promenade von Erstsemester über'n Haufen gerannt worden. Hat mich gemein beschimpft.\\ 
%\\
06.45 Bude mühevoll erreicht. Insgesamt \unit[15]{\officialeuro}  ausgegeben. Mehr hatte die Kleine nicht dabei.\\ 
%\\
07.05 Ich schlucke schnell noch ein paar Alkas und schalte kurz das Radio ein. Stimme des Sprechers: "Guten Morgen liebe Zuhörer, gute Nacht liebe Studenten."\\
\end{verse}

\newpage
\section{Gängige Abkürzungen im Uni-Dschungel}

\begin{longtable}{p{0.2\textwidth} p{0.7\textwidth}}
  \textbf{AOR} & Akademischer Oberrat (der „Mittelbau“)\\
  \textbf{AR} & Akademischer Rat\\
  \textbf{AStA} & Allgemeiner Studierendenausschuss\\
  \textbf{ASV} & Ausländische Studierendenvertretung\\
  \textbf{AVZ} & Allgemeines Verfügungszentrum\\
  \textbf{Ba / BSc} & Bachelor / Bachelor of Science\\
  \textbf{Bib} & Bibliothek\\
  \textbf{c.t.} & cum tempore = akademisches Viertel (\unit[9]{Uhr} c.t. = \unit[9:15]{Uhr})\\
  \textbf{Dipl.} & Diplom\\
  \textbf{DSW} & Deutsches Studentenwerk\\
  \textbf{ECTS} & European Credit Transfer System\\
  \textbf{EGEA} & European Geographic Association\\
  \textbf{ERASMUS} & Europäisches Austauschprogramm\\
  \textbf{ESG} & Evangelische Studierendengemeinde\\
  \textbf{Exk.} & Exkursion\\
  \textbf{FB} & Fachbereich\\
  \textbf{FBR} & Fachbereichsrat\\
  \textbf{FH} & Fachhochschule\\
  \textbf{FK} & Fachschaftenkonferenz\\
  \textbf{FO} & Frontoffice\\
  \textbf{FP} & Fachprüfung (meistens mündlich, kann schriftlich sein)\\
  \textbf{FS} & Fachschaft \tiny{(oder Fachsemester)}\\ 
  \textbf{FSR} & Fachschaftsrat\\
  \textbf{FSV} & Fachschaftsvertretung\\
  \textbf{fsz} & Freier Zusammenschluss von Studierenden (\underline{der} Dachverband)\\       
  \textbf{GeLaGe} & Geographisch-Landschaftsökologische Gemeinschaftsliste\\ 
  \textbf{GelPr.} & Geländepraktikum\\
  \textbf{Geofs} & Fachschaft Geoinformatik\\
  \textbf{Geogr.} & Geographie oder Geograph/Geographin\\
  \textbf{GeoLök} & Fachschaft Geographie u. Landschaftsökologie\\
  \textbf{GHR} & Grund-, Haupt-, Realschule\\
  \textbf{GPI} & Geologisch-Paläontologisches Institut (Corrensstr.)\\
  \textbf{Gym/Ges} & Gymnasium/ Gesamtschule\\
  \textbf{HD} & Hochschuldozent\\
  \textbf{HISLSF} & siehe LSF\\
  \textbf{Hiwi} & Hilfswissenschaftler; Hilfskraft\\
  \textbf{HoMaLa} & Horstmarer Landweg (Studentenwohnheime)\\
  \textbf{HRG} & Hochschulrahmengesetz\\
  \textbf{HS} & Hörsaal (Heisenbergstr.)\\
  \textbf{HSP} & Hochschulsport\\
  \textbf{IfDG} & Institut für Didaktik der Geographie (Heisenbergstr.)\\
  \textbf{IfG} & Institut für Geographie (Heisenbergstr.)\\
  \textbf{ifgi} & Institut für Geoinformatik (Heisenbergstr.)\\
  \textbf{ILök} & Institut für Landschaftsökologie (Heisenbergstr.)\\
  \textbf{IO} & International Office\\
  \textbf{IVV} & Informationsverarbeitungsversorgung\\
  \textbf{KHG} & Katholische Hochschulgemeinde (am Kardinal-von-Galen-Ring)\\
  \textbf{Kom-Voz} & Kommentiertes Vorlesungsverzeichnis\\
  \textbf{KLSA} & Kommission für Lehre und studentische Angelegenheiten\\
  \textbf{KSHG} & Katholische Studierenden- und Hochschulgemeinde (Frauenstr.)\\
  \textbf{LA} & Lehramt\\
  \textbf{LABG} & Lehrerausbildungsgesetz\\
  \textbf{LN} & Leistungsnachweis, auch "`Schein"' genannt\\
  \textbf{LP} & Leistungspunkt, entspricht 30 Arbeitsstunden, sieht ECTS\\ 
  \textbf{LPO} & Lehramtsprüfungsordnung\\
  \textbf{Lök} & Landschaftsökologie\\
  \textbf{LSF / HISLSF} & Elektronisches Vorlesungsverzeichnis der Uni Münster „Lehre, Studium, Forschung“ (s.o.)\\
  \textbf{Ma / MSc} & Master / Master of Science\\
  \textbf{MAP} & Modulabschlussprüfung\\
  \textbf{N.N.} & Nomen Nominandum (der Name des Dozierenden wird noch bekannt gegeben)\\
  \textbf{Prakt.} & Praktikum\\
  \textbf{Prof.} & Professor\\
  \textbf{Proj.} & Projekt\\
  \textbf{QisPos} & Elektronisches System für alle Bachelor zur Anmeldung und Registrierung (s.o.)\\
  \textbf{RHW} & Rudolf-Harbig-Weg (Studiwohnheime)\\
  \textbf{Sepl} & Seminarplatzvergabe (online)\\
  \textbf{S I/II} & Sekundarstufe I/II\\
  \textbf{SP / StuPa} & Studierendenparlament\\ 
  \textbf{SpSt} & Schulpraktische Studien (Büro Scharnhorststr. 100 / Platz der weißen Rose)\\
  \textbf{StuPa} & siehe SP\\
  \textbf{SoSe} & Sommersemester (bitte \textbf{keine} andere Abkürzung verwenden)\\
  \textbf{s.t.}	& sine tempore = ohne akademisches Viertel pünktlich (\unit[9]{Uhr} s.t. = \unit[9:00]{Uhr})\\
  \textbf{StudLab} & Studenten Labor = Computerraum\\
  \textbf{SWS} & Semesterwochenstunden\\
  \textbf{TN} & Teilnahmenachweis\\
  \textbf{Ü\,/\,ÜB} & Übung\\
  \textbf{UB\,/\,ULB} & Universitäts- und Landesbibliothek (Krummer Timpen)\\
  \textbf{Vorl.\,/\,V} & Vorlesung\\
  \textbf{VV} & Vollversammlung oder Vorlesungsverzeichnis\\
  \textbf{WS} & Wintersemester\\
  \textbf{ZDM/MD} & Zentrum für Digitale Medien und Mediendidaktik\\
  \textbf{ZSB} & Zentrale Studienberatung (Schloßplatz)\\
\end{longtable} 

\newpage
\section{Die studentische Selbstverwaltung}
\textbf{Studierendenschaft}
\begin{itemize}
 \item alle Studierenden der Universität
 \item wählen jedes Wintersemester VertreterInnen ins SP
\end{itemize}

\textbf{Studierendenparlament (SP)}
\begin{itemize}
  \item hat 31 Mitglieder, VertreterInnen verschiedener hochschulpolitischer Listen, werden für ein Jahr gewählt
  \item wählt aus seiner Mitte den Vorsitz für das SP
  \item wählt AStA-Vorstand und die ordentlichen AStA-Referate
  \item bestätigt die autonomen Referate, die von ihrer jeweiligen Interessengruppe gewählt werden
  \item oberstes Beschlussfassendes Gremium der Studierendenschaft
\end{itemize}

\textbf{Allgemeiner Studierenden Ausschuss (AStA)}
\begin{itemize}
  \item Exekutivorgan der Studierendenschaft (wie Merkel mit ihren Bundesministerien)
  \item besteht aus Vorstand und Referaten:
    \begin{enumerate}
      \item ordentliche Referate \\ (Finanz-, Hochschulpolitik-, Wohn-, Sozial"~, Öffentlichkeits-, Ökologie-, Kultur-, Frieden/Internationalismus- Referate)
      \item autonome Referate \\ (Frauen-, Lesben-, Schwulen- und Behindertenreferat)
      \item halbautonome Referate (Fachschaftenkonferenz)
    \end{enumerate}
  \item auskunftspflichtig gegenüber dem SP
\end{itemize}

\textbf{Fachschaft}
\begin{itemize}
  \item alle Studierenden eines Fachbereiches
  \item wählen jedes Sommersemester die Fachschaftsvertretung                                                  
\end{itemize}

\textbf{Fachschaftsvertretung (FSV)}
\begin{itemize}
  \item entspricht strukturell dem SP, auch hier können verschiedene Listen zur Wahl antreten
  \item wählt eigenen Vorsitz
  \item beschließt Satzung der Fachschaft
  \item wählt den FSR
\end{itemize}

\textbf{Fachschaftsrat (FSR)} 
\begin{itemize}
  \item ausführendes Organ der Fachschaft
  \item Koordinierung der studentischen Politik am Fachbereich
  \item Veröffentlichung von Infos aus dem Fachbereich, dem universitären und dem überregionalen Bereich
  \item Engagement in den Gremien des Fachbereichs
  \item ErstsemesterInnenarbeit, Serviceleistungen
  \item schickt VertreterIn zur Fachschaftenkonferenz
\end{itemize}

\textbf{Fachschaftenkonferenz (FK)} 
\begin{itemize}
  \item Organ, in dem VertreterInnen aller Fachschaften der Universität Münster zusammenkommen
  \item Bindeglied zwischen AStA, SP auf der einen und den Fachschaften auf der anderen Seite
  \item wählt FK-ReferentIn
  \item beratende Mitglieder sind VertreterInnen aus Uni-Kommissionen und dem AStA
  \item FK-ReferentInnen sind dem AStA auskunftspflichtig und der FK rechenschaftspflichtig
\end{itemize}

\section{Gremien der Universität}

\textbf{Institutsvorstand}
\begin{itemize}
 \item Professoren eines Institutes gehören ihm automatisch an
 \item auf jeden vierten Professor  kommt ein Mitglied der anderen Statusgruppen
 \item Umsetzung der Prüfungsordnungen in Studienordnungen 
  \item studentischen VertreterInnen werden von den studentischen Mitgliedern im FBR gewählt
  \item wählt aus der Gruppe der Professoren den/ die Direktor/in
\end{itemize}

\textbf{FBR (Fachbereichsrat)}
\begin{itemize}
  \item höchstes Beschlussfassendes Gremium des Fachbereiches
  \item entscheidet in allen Belangen des Fachbereiches: Berufungen, Finanzen, Lehrangebot
  \item Vorsitz hat der/die DekanIn; wird aus der Gruppe der Profs gewählt
  \item Mitglieder werden von jeweiliger Statusgruppe gewählt
  \item hat mehrere Ausschüsse, zum Beispiel:
    \begin{itemize}
	\item \textbf{AFWN} (Ausschuss für Forschung und wissenschaftlichen \linebreak Nachwuchs)
	\item \textbf{ALsA} ( Ausschuss für Lehre und studentische Angelegenheiten)
	\item \textbf{DPA} (Diplomprüfungsausschuss)
	\item Prüfungsausschuss LA SI/II
	\item Berufungskommissionen
	\item Bachelor/Master-Ausschuss
    \end{itemize}
\end{itemize}

\textbf{Dekan}
\begin{itemize}
 \item vollzieht Promotionen, Habilitationen
  \item hat Eilkompetenz in wichtigen Angelegenheiten
  \item lädt zu den Sitzungen des Fachbereichrates ein
  \item hat Rederecht im Senat
  \item aktuelle Dekanatsregelung im Fachbereich Geowissenschaften:
      \begin{itemize}
	\item 1 Dekan
	\item 2 Prodekane: Finanzdekan und Studiendekan (den Job des Studiendekans kann auch ein Student machen!!!)
      \end{itemize}
\end{itemize}

\textbf{Gleichstellungsbeauftragte/r des Fachbereichs}
\begin{itemize}
 \item FBR stellt eine/n Gleichstellungsbeauftragte/n
  \item offiziell: ein/e Gleichstellungsbeauftragte/n und bis zu drei StellvertreterInnen -- meistens aber eher ein/e studentische/r, ein/e nichtwissenschaftliche/r sowie ein/e wissenschaftliche/r Gleichstellungsbeauftragte/r
  \item für alle Belange zuständig, die Gleichstellung innerhalb des Fachbereiches betreffen
  \item Rede- und Teilnahmerecht in allen Gremien des Fachbereiches, soweit es um die Belange der Gleichstellung geht
\end{itemize}

\textbf{Fakultätsrat}
\begin{itemize}
 \item setzt sich aus Mitgliedern der einzelnen Fachbereichrate zusammen
  \item Vorsitz hat FakultätsdekanIn: Gewählt aus Gruppe der Profs
  \item beschäftigt sich mit fächerübergreifenden Angelegenheiten
\end{itemize}

\textbf{Senat}
\begin{itemize}
 \item hat am 11.7.07 bzw. am 7.2.08 nahezu alle bedeutenden Entscheidungskompetenzen an den Hochschulrat abgegeben
\end{itemize}

\textbf{Hochschulrat}
\begin{itemize}
 \item höchstes Beschlussfassendes Gremium der Universität
  \item Entscheidungen über Verteilung der Stellen und Finanzen, Einrichtung/Aufhebung von Fachbereichen
  \item Beschlüsse über Satzungen und Ordnungen der Universität
  \item Anträge an den Konvent
  \item besteht aus 5 uni-externen und 3 uni-internen Mitgliedern
  \item soll einem Aufsichtsrat entsprechen
  \item wählt Hochschulleitung, stimmt über Hochschulentwicklungs- \linebreak und Wirtschaftsplan ab und kann Einrichtung und Schließung von Studiengängen beschließen
  \item tagt nicht öffentlich
  \item weiteres Beispiel für den Wandel an den Hochschulen zu Kaderschmieden der Industrie, da nun eine Mehrheit aus Wirtschaftsvertretern gänzlich ohne Mitsprache des Mittelbaus oder der Studierenden, die wichtigsten Entscheidungen der Uni trifft
\end{itemize}

\textbf{Rektorat}
\begin{itemize}
 \item hat Rederecht im Senat
  \item besteht aus ProrektorInnen, KanzlerIn und RektorIn
  \item bereitet Senatssitzungen vor
  \item dem Senat gegenüber rechenschaftspflichtig
  \item entscheidet in Verwaltungsangelegenheiten
  \item ProrektorInnen haben ständigen Vorsitz in Kommissionen des Senats
  \item RektorIn beruft Senatssitzungen ein und führt dessen Beschlüsse aus
\end{itemize}

\newpage

\section{Unsere Professoren}
\begin{small}
\begin{longtable}{p{0.4\columnwidth} p{0.3\columnwidth} p{0.2\columnwidth}}
  Name & eMail und Telefon & Institut\\ \hline \hline
  Prof.\,Dr.\,Michael Hemmer & \url{michael.hemmer} \newline Tel.:\,83--39365 & Didaktik Geographie\\
  Prof.\,Dr.\,Gabriele Schrüfer & \url{gabriele.schruefer} \newline Tel.:\,83--39349 & Didaktik Geographie\\  \hline
  Prof.\,Dr.\,Paul Reuber & \url{p.reuber} \newline Tel.:\,83--30035 & Geographie\\
  Prof.\,Dr.\,Gerald Wood & \url{geosek} \newline Tel.:\,83--30026 & Geographie\\
  Prof.\,Dr.\,Samuel Mössner & \url{moessner} \newline Tel.:\,83--33922 & Geographie\\\hline
  Prof.\,Dr.\,Angela Schwering & \url{angela.schwering} \newline Tel.:\,83--33059 & Geoinformatik\\
  Prof.\,Dr.\,Edzer Pebesma & \url{edzer.pebesma} \newline Tel.:\,83--33081 & Geoinformatik\\ 
  Prof.\,Dr.\,Christian Kray & \url{c.kray} \newline Tel.:\,83--33073 &  Geoinformatik \\ \hline
  Prof.\,Dr.\newline Tillmann Buttschardt & \url{tillmann.buttschardt} \newline Tel.:\,83--30104 & Lök\\
  Prof.\,Dr.\,Norbert Hölzel & \url{norbert.hoelzel} \newline Tel.:\,83--33994 & Lök\\
  Prof.\,Dr.\,Otto Klemm & \url{otto.klemm} \newline Tel.:\,83--33921 & Lök\\
  Prof.\,Dr.\,Christoph Scherber & \url{christoph.scherber} \newline Tel.:\,83--33996 & Lök\\
  Prof.\,Dr.\,Andreas Schulte & \url{andreas.schulte} \newline \url{@wald-zentrum.de} \newline Tel.:\,83--30121 & Lök\\ 
 \hline
  \multicolumn{3}{l}{Bei den \url{eMail Adressen} ist jeweils} \\
  \multicolumn{3}{l}{``@uni-muenster.de'' zu ergänzen.}\\
\end{longtable}
\end{small}

%\newpage

%\textbf{StudienberaterInnen:}\\ \\
%\begin{tabular}{p{0.3\columnwidth} p{0.7\columnwidth}}
%Geographie & Dr. Christoph Scheuplein \newline \url{christoph.scheuplein}|Tel.:\,83--33925\\
%&\\
%Geographie \newline (2-Fach Bacherlor) & Prof.\,Dr.\,Gerald Woold \newline \url{geosek}|Tel.:\,83--30025\\
%&\\
%Geographie \newline (Lehrämter) & Dipl.\,Geogr.\,Katja Wrenger \newline \url{katja.wrenger}|Tel.:\,83--39364\\
%&\\
%Geoinformatik & Prof.\,Dr.\,Angela Schwering \newline \url{angela.schwering}|Tel.:\,83--33059\\
%&\\
%Lök & Dr. Andreas Vogel \newline \url{voghild}|Tel.:\,83--33698\\ \hline
%\multicolumn{2}{l}{Bei den \url{eMail Adressen} ist jeweils}\\
%\multicolumn{2}{l}{``@uni-muenster.de'' zu ergänzen!!!}\\
%\end{tabular} 
